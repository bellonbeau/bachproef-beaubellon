%%=============================================================================
%% Inleiding
%%=============================================================================

\chapter{\IfLanguageName{dutch}{Inleiding}{Introduction}}
\label{ch:inleiding}

Drones zijn al sinds enkele decennia ingezet met vooral militaire doeleinden, maar sinds enkele jaren hebben deze ook hun plaats gevonden voor maatschappelijke doelen. Meer bepaald worden deze steeds meer gebruikt in gebieden zoals in de vastgoed, recreatief of bij calamiteiten in rampgebieden. In dit onderzoek wordt er op zoek gegaan op welke manier drones een hulpmiddel zijn geweest tijdens en na de bosbranden in Australië en op welke wijze dit specifiek is gebeurt. Er wordt in het bijzonder onderzocht hoe deze drones via het maken van foto's, helpen bij het blussen van branden en het identificeren en herkennen van bepaalde dieren en plantensoorten een belangrijke rol hebben gespeeld. Anderzijds zal er ook uitbundig besproken worden wat de technische specificaties en functionaliteiten van deze drones juist moeten omvatten, alsook de beveiliging zodat deze niet kunnen misbruikt worden. Er wordt ook dieper ingegaan op wat er allemaal met deze data kan en moet gebeuren zodanig dat deze bruikbaar is en in de toekomst opnieuw relevant kan zijn. 

In een levensbedreigende situatie is het vaak zo dat elke seconde telt en dan zou de hulp van een drone, autonoom of manueel, wel eens het verschil kunnen maken. Het doel van dit onderzoek is dan ook om na te gaan in welke mate de drones hulpverlening hebben kunnen bieden tijdens en na deze bosbranden alsook een reflectie op welke manier dit misschien beter had kunnen verlopen. Er zal dus voornamelijk uitgeweid worden welke technieken de drones hier hebben toegepast en deze uitbundig uit te leggen met nodige technische specificaties en functionaliteiten.


\section{\IfLanguageName{dutch}{Probleemstelling}{Problem Statement}}
\label{sec:probleemstelling}

In een rampgebied is er vaak een tekort aan hulpverlening wegens het gebrek aan voldoende opgeleide mensen die hiervoor bekwaam genoeg is. Een drone kan hierbij een extra hulpmiddel zijn die in samenwerking met de hulpverleners een grote meerwaarde kan bieden. Het manueel in kaart brengen van verwoesting door bosbranden is niet evident, door het gebruik van een drone kan dit veel efficiënter en veiliger gebeuren.

Omdat dit onderzoek de fundamentele hulp bespreekt die drones kunnen bieden bij een calamiteit, kan dit als basis dienen voor hulp bij andere calamiteiten in rampgebieden.

Dit onderzoek kan als dusdanig een meerwaarde zijn voor brandweer, politie en bedrijven die zich specialiseren in drones.

\section{\IfLanguageName{dutch}{Onderzoeksvraag}{Research question}}
\label{sec:onderzoeksvraag}

\subsection{\IfLanguageName{dutch}{Hoofdonderzoeksvraag}{Research question}}
\label{sec:hoofdonderzoeksvraag}

Dit onderzoek zal zich toespitsen op de verschillende gebieden waarin drones hulp hebben kunnen bieden. Hierbij gaat het over het identificeren en redden van dieren, het in kaart brengen van de schade aan gebieden door de bosbranden, het helpen van blussen bij branden. Hieruit volgt de hoofdonderszoeksvraag.

\begin{itemize}
\item Op welke manier zijn drones een hulpmiddel geweest tijdens de bosbranden in Australië?
\end{itemize}{}

\subsection{\IfLanguageName{dutch}{Deelonderzoeksvragen}{Research question}}
\label{sec:deelonderzoeksvragen}

Omdat dit onderzoek zich op verschillende facetten richt zal er naast de hoofd onderzoeksvraag nog enkele deel onderzoeksvragen zijn.

\begin{itemize}
\item Wat gebeurt er met de data die door drones wordt verzameld en welke technieken hanteren ze om deze data te bemachtigen?
\item Wat zijn de beperkingen die drones met zich meedragen?
\item Wat zijn mogelijke alternatieven voor een drone?
\end{itemize}{}

\section{\IfLanguageName{dutch}{Onderzoeksdoelstelling}{Research objective}}
\label{sec:onderzoeksdoelstelling}

Het primaire doel van dit onderzoek is om een concreet beeld te schetsen van de manier waarop drones tijdens de bosbranden in Australië een hulpmiddel zijn geweest. Het tweede doel van dit onderzoek gaat dieper in op de technische aspecten van die hulp. Hoe de drones data verzamelen en hoe deze gebruikt wordt. Er wordt gesproken over de verschillende technieken die worden gebruikt om de data die drones verzamelen bruikbaar te maken. Vervolgens wordt er gekeken naar de beperkingen waarover drones beschikken op technisch en mechanisch vlak. Er wordt besproken of een drone voor zulke omstandigheden een goede oplossing biedt en er wordt vergeleken met alternatieven.

\section{\IfLanguageName{dutch}{Opzet van deze bachelorproef}{Structure of this bachelor thesis}}
\label{sec:opzet-bachelorproef}

% Het is gebruikelijk aan het einde van de inleiding een overzicht te
% geven van de opbouw van de rest van de tekst. Deze sectie bevat al een aanzet
% die je kan aanvullen/aanpassen in functie van je eigen tekst.

De rest van deze bachelorproef is als volgt opgebouwd:

In Hoofdstuk~\ref{ch:stand-van-zaken} wordt een overzicht gegeven van de stand van zaken binnen het onderzoeksdomein, op basis van een literatuurstudie.

In Hoofdstuk~\ref{ch:methodologie} wordt de methodologie toegelicht en worden de gebruikte onderzoekstechnieken besproken om een antwoord te kunnen formuleren op de onderzoeksvragen.

% TODO: Vul hier aan voor je eigen hoofstukken, één of twee zinnen per hoofdstuk

In Hoofdstuk~\ref{ch:conclusie}, tenslotte, wordt de conclusie gegeven en een antwoord geformuleerd op de onderzoeksvragen. Daarbij wordt ook een aanzet gegeven voor toekomstig onderzoek binnen dit domein.