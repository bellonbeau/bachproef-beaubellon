%===============================================================================
% LaTeX sjabloon voor de bachelorproef toegepaste informatica aan HOGENT
% Meer info op https://github.com/HoGentTIN/bachproef-latex-sjabloon
%===============================================================================

\documentclass{bachproef-tin}

\usepackage{hogent-thesis-titlepage} % Titelpagina conform aan HOGENT huisstijl

%%---------- Documenteigenschappen ---------------------------------------------
% TODO: Vul dit aan met je eigen info:

% De titel van het rapport/bachelorproef
\title{Drones als hulpmiddel bij de bosbranden in Australië}

% Je eigen naam
\author{Beau Bellon}

% De naam van je promotor (lector van de opleiding)
\promotor{Guy Dekoning}

% De naam van je co-promotor. Als je promotor ook je opdrachtgever is en je
% dus ook inhoudelijk begeleidt (en enkel dan!), mag je dit leeg laten.
\copromotor{David Van Impe}

% Indien je bachelorproef in opdracht van/in samenwerking met een bedrijf of
% externe organisatie geschreven is, geef je hier de naam. Zoniet laat je dit
% zoals het is.
\instelling{---}

% Academiejaar
\academiejaar{2019-2020}

% Examenperiode
%  - 1e semester = 1e examenperiode => 1
%  - 2e semester = 2e examenperiode => 2
%  - tweede zit  = 3e examenperiode => 3
\examenperiode{2}

%===============================================================================
% Inhoud document
%===============================================================================

\begin{document}

%---------- Taalselectie -------------------------------------------------------
% Als je je bachelorproef in het Engels schrijft, haal dan onderstaande regel
% uit commentaar. Let op: de tekst op de voorkaft blijft in het Nederlands, en
% dat is ook de bedoeling!

%\selectlanguage{english}

%---------- Titelblad ----------------------------------------------------------
\inserttitlepage

%---------- Samenvatting, voorwoord --------------------------------------------
\usechapterimagefalse
%%=============================================================================
%% Voorwoord
%%=============================================================================

\chapter*{\IfLanguageName{dutch}{Woord vooraf}{Preface}}
\label{ch:voorwoord}

%% TODO:
%% Het voorwoord is het enige deel van de bachelorproef waar je vanuit je
%% eigen standpunt (``ik-vorm'') mag schrijven. Je kan hier bv. motiveren
%% waarom jij het onderwerp wil bespreken.
%% Vergeet ook niet te bedanken wie je geholpen/gesteund/... heeft


%%=============================================================================
%% Samenvatting
%%=============================================================================

% TODO: De "abstract" of samenvatting is een kernachtige (~ 1 blz. voor een
% thesis) synthese van het document.
%
% Deze aspecten moeten zeker aan bod komen:
% - Context: waarom is dit werk belangrijk?
% - Nood: waarom moest dit onderzocht worden?
% - Taak: wat heb je precies gedaan?
% - Object: wat staat in dit document geschreven?
% - Resultaat: wat was het resultaat?
% - Conclusie: wat is/zijn de belangrijkste conclusie(s)?
% - Perspectief: blijven er nog vragen open die in de toekomst nog kunnen
%    onderzocht worden? Wat is een mogelijk vervolg voor jouw onderzoek?
%
% LET OP! Een samenvatting is GEEN voorwoord!

%%---------- Nederlandse samenvatting -----------------------------------------
%
% TODO: Als je je bachelorproef in het Engels schrijft, moet je eerst een
% Nederlandse samenvatting invoegen. Haal daarvoor onderstaande code uit
% commentaar.
% Wie zijn bachelorproef in het Nederlands schrijft, kan dit negeren, de inhoud
% wordt niet in het document ingevoegd.

\IfLanguageName{english}{%
\selectlanguage{dutch}
\chapter*{Samenvatting}
\lipsum[1-4]
\selectlanguage{english}
}{}

%%---------- Samenvatting -----------------------------------------------------
% De samenvatting in de hoofdtaal van het document

\chapter*{\IfLanguageName{dutch}{Samenvatting}{Abstract}}

\lipsum[1-4]


%---------- Inhoudstafel -------------------------------------------------------
\pagestyle{empty} % Geen hoofding
\tableofcontents  % Voeg de inhoudstafel toe
\cleardoublepage  % Zorg dat volgende hoofstuk op een oneven pagina begint
\pagestyle{fancy} % Zet hoofding opnieuw aan

%---------- Lijst figuren, afkortingen, ... ------------------------------------

% Indien gewenst kan je hier een lijst van figuren/tabellen opgeven. Geef in
% dat geval je figuren/tabellen altijd een korte beschrijving:
%
%  \caption[korte beschrijving]{uitgebreide beschrijving}
%
% De korte beschrijving wordt gebruikt voor deze lijst, de uitgebreide staat bij
% de figuur of tabel zelf.

\listoffigures
\listoftables

% Als je een lijst van afkortingen of termen wil toevoegen, dan hoort die
% hier thuis. Gebruik bijvoorbeeld de ``glossaries'' package.
% https://www.overleaf.com/learn/latex/Glossaries

%---------- Kern ---------------------------------------------------------------

% De eerste hoofdstukken van een bachelorproef zijn meestal een inleiding op
% het onderwerp, literatuurstudie en verantwoording methodologie.
% Aarzel niet om een meer beschrijvende titel aan deze hoofstukken te geven of
% om bijvoorbeeld de inleiding en/of stand van zaken over meerdere hoofdstukken
% te verspreiden!

%%=============================================================================
%% Inleiding
%%=============================================================================

\chapter{\IfLanguageName{dutch}{Inleiding}{Introduction}}
\label{ch:inleiding}

Drones zijn al sinds enkele decennia ingezet met vooral militaire doeleinden, maar sinds enkele jaren hebben deze ook hun plaats gevonden voor maatschappelijke doelen. Meer bepaald worden deze steeds meer gebruikt in gebieden zoals in de vastgoed, recreatief of bij calamiteiten in rampgebieden. In dit onderzoek wordt er op zoek gegaan op welke manier drones een hulpmiddel zijn geweest tijdens en na de bosbranden in Australië en op welke wijze dit specifiek is gebeurt. Er wordt in het bijzonder onderzocht hoe deze drones via het maken van foto's, helpen bij het blussen van branden en het identificeren en herkennen van bepaalde dieren en plantensoorten een belangrijke rol hebben gespeeld. Anderzijds zal er ook uitbundig besproken worden wat de technische specificaties en functionaliteiten van deze drones juist moeten omvatten, alsook de beveiliging zodat deze niet kunnen misbruikt worden. Er wordt ook dieper ingegaan op wat er allemaal met deze data kan en moet gebeuren zodanig dat deze bruikbaar is en in de toekomst opnieuw relevant kan zijn. 

In een levensbedreigende situatie is het vaak zo dat elke seconde telt en dan zou de hulp van een drone, autonoom of manueel, wel eens het verschil kunnen maken. Het doel van dit onderzoek is dan ook om na te gaan in welke mate de drones hulpverlening hebben kunnen bieden tijdens en na deze bosbranden alsook een reflectie op welke manier dit misschien beter had kunnen verlopen. Er zal dus voornamelijk uitgeweid worden welke technieken de drones hier hebben toegepast en deze uitbundig uit te leggen met nodige technische specificaties en functionaliteiten.


\section{\IfLanguageName{dutch}{Probleemstelling}{Problem Statement}}
\label{sec:probleemstelling}

In een rampgebied is er vaak een tekort aan hulpverlening wegens het gebrek aan voldoende opgeleide mensen die hiervoor bekwaam genoeg is. Een drone kan hierbij een extra hulpmiddel zijn die in samenwerking met de hulpverleners een grote meerwaarde kan bieden. Het manueel in kaart brengen van verwoesting door bosbranden is niet evident, door het gebruik van een drone kan dit veel efficiënter en veiliger gebeuren.

Omdat dit onderzoek de fundamentele hulp bespreekt die drones kunnen bieden bij een calamiteit, kan dit als basis dienen voor hulp bij andere calamiteiten in rampgebieden.

Dit onderzoek kan als dusdanig een meerwaarde zijn voor brandweer, politie en bedrijven die zich specialiseren in drones.

\section{\IfLanguageName{dutch}{Onderzoeksvraag}{Research question}}
\label{sec:onderzoeksvraag}

\subsection{\IfLanguageName{dutch}{Hoofdonderzoeksvraag}{Research question}}
\label{sec:hoofdonderzoeksvraag}

Dit onderzoek zal zich toespitsen op de verschillende gebieden waarin drones hulp hebben kunnen bieden. Hierbij gaat het over het identificeren en redden van dieren, het in kaart brengen van de schade aan gebieden door de bosbranden, het helpen van blussen bij branden. Hieruit volgt de hoofdonderszoeksvraag.

\begin{itemize}
\item Op welke manier zijn drones een hulpmiddel geweest tijdens de bosbranden in Australië?
\end{itemize}{}

\subsection{\IfLanguageName{dutch}{Deelonderzoeksvragen}{Research question}}
\label{sec:deelonderzoeksvragen}

Omdat dit onderzoek zich op verschillende facetten richt zal er naast de hoofd onderzoeksvraag nog enkele deel onderzoeksvragen zijn.

\begin{itemize}
\item Wat gebeurt er met de data die door drones wordt verzameld en welke technieken hanteren ze om deze data te bemachtigen?
\item Wat zijn de beperkingen die drones met zich meedragen?
\item Wat zijn mogelijke alternatieven voor een drone?
\end{itemize}{}

\section{\IfLanguageName{dutch}{Onderzoeksdoelstelling}{Research objective}}
\label{sec:onderzoeksdoelstelling}

Het primaire doel van dit onderzoek is om een concreet beeld te schetsen van de manier waarop drones tijdens de bosbranden in Australië een hulpmiddel zijn geweest. Het tweede doel van dit onderzoek gaat dieper in op de technische aspecten van die hulp. Hoe de drones data verzamelen en hoe deze gebruikt wordt. Er wordt gesproken over de verschillende technieken die worden gebruikt om de data die drones verzamelen bruikbaar te maken. Vervolgens wordt er gekeken naar de beperkingen waarover drones beschikken op technisch en mechanisch vlak. Er wordt besproken of een drone voor zulke omstandigheden een goede oplossing biedt en er wordt vergeleken met alternatieven.

\section{\IfLanguageName{dutch}{Opzet van deze bachelorproef}{Structure of this bachelor thesis}}
\label{sec:opzet-bachelorproef}

% Het is gebruikelijk aan het einde van de inleiding een overzicht te
% geven van de opbouw van de rest van de tekst. Deze sectie bevat al een aanzet
% die je kan aanvullen/aanpassen in functie van je eigen tekst.

De rest van deze bachelorproef is als volgt opgebouwd:

In Hoofdstuk~\ref{ch:stand-van-zaken} wordt een overzicht gegeven van de stand van zaken binnen het onderzoeksdomein, op basis van een literatuurstudie.

In Hoofdstuk~\ref{ch:methodologie} wordt de methodologie toegelicht en worden de gebruikte onderzoekstechnieken besproken om een antwoord te kunnen formuleren op de onderzoeksvragen.

% TODO: Vul hier aan voor je eigen hoofstukken, één of twee zinnen per hoofdstuk

In Hoofdstuk~\ref{ch:conclusie}, tenslotte, wordt de conclusie gegeven en een antwoord geformuleerd op de onderzoeksvragen. Daarbij wordt ook een aanzet gegeven voor toekomstig onderzoek binnen dit domein.
\chapter{\IfLanguageName{dutch}{Stand van zaken}{State of the art}}
\label{ch:stand-van-zaken}

% Tip: Begin elk hoofdstuk met een paragraaf inleiding die beschrijft hoe
% dit hoofdstuk past binnen het geheel van de bachelorproef. Geef in het
% bijzonder aan wat de link is met het vorige en volgende hoofdstuk.

% Pas na deze inleidende paragraaf komt de eerste sectiehoofding.

Dit hoofdstuk bevat je literatuurstudie. De inhoud gaat verder op de inleiding, maar zal het onderwerp van de bachelorproef *diepgaand* uitspitten. De bedoeling is dat de lezer na lezing van dit hoofdstuk helemaal op de hoogte is van de huidige stand van zaken (state-of-the-art) in het onderzoeksdomein. Iemand die niet vertrouwd is met het onderwerp, weet nu voldoende om de rest van het verhaal te kunnen volgen, zonder dat die er nog andere informatie moet over opzoeken \autocite{Pollefliet2011}.

Je verwijst bij elke bewering die je doet, vakterm die je introduceert, enz. naar je bronnen. In \LaTeX{} kan dat met het commando \texttt{$\backslash${textcite\{\}}} of \texttt{$\backslash${autocite\{\}}}. Als argument van het commando geef je de ``sleutel'' van een ``record'' in een bibliografische databank in het Bib\LaTeX{}-formaat (een tekstbestand). Als je expliciet naar de auteur verwijst in de zin, gebruik je \texttt{$\backslash${}textcite\{\}}.
Soms wil je de auteur niet expliciet vernoemen, dan gebruik je \texttt{$\backslash${}autocite\{\}}. In de volgende paragraaf een voorbeeld van elk.

\textcite{Knuth1998} schreef een van de standaardwerken over sorteer- en zoekalgoritmen. Experten zijn het erover eens dat cloud computing een interessante opportuniteit vormen, zowel voor gebruikers als voor dienstverleners op vlak van informatietechnologie~\autocite{Creeger2009}.

\lipsum[7-20]

``%%=============================================================================
%% Methodologie
%%=============================================================================

\chapter{\IfLanguageName{dutch}{Methodologie}{Methodology}}
\label{ch:methodologie}

%% TODO: Hoe ben je te werk gegaan? Verdeel je onderzoek in grote fasen, en
%% licht in elke fase toe welke stappen je gevolgd hebt. Verantwoord waarom je
%% op deze manier te werk gegaan bent. Je moet kunnen aantonen dat je de best
%% mogelijke manier toegepast hebt om een antwoord te vinden op de
%% onderzoeksvraag.

\lipsum[21-25]



\input{DronesHulpmiddel}
\chapter{Data}
In dit onderdeel zal de technische kant meer besproken worden, op welke manier data verkregen wordt en wat er met deze data gebeurt. Het is de bedoeling dat hierin de volledige architectuur van een drone wordt uitgelegd gaande van data verzameling tot bruikbaarheid. Stappen die doorlopen zullen worden zijn:
\begin{itemize}
    \item Hoe verkrijgt de drone zijn data?
    \item Met welke zaken is een drone voor een specifieke 'missie' uitgerust?
    \item Hoe slaat men deze data op, databases, datacenter, netwerk...
    \item Applicaties die vaak worden gebruikt in zulke situaties 
    \item Data mining, AI technieken uitleggen
    \item Leren uit data voor toekomstige calamiteiten...
\end{itemize}

\chapter{Beperkingen en alternatieven}
In dit (laatste) hoofdstuk zal er gekeken worden naar alternatieven die worden gebruikt tegenover een drone. Dit zal gaan over satellieten en andere zaken... Er wordt ook gekeken naar de beperkingen die een drone met zich mee brengt, gaande voornamelijk op technisch/mechanisch/communicatie vlak. 

% Voeg hier je eigen hoofdstukken toe die de ``corpus'' van je bachelorproef
% vormen. De structuur en titels hangen af van je eigen onderzoek. Je kan bv.
% elke fase in je onderzoek in een apart hoofdstuk bespreken.

%\input{...}
%\input{...}
%...

%%=============================================================================
%% Conclusie
%%=============================================================================

\chapter{Conclusie}
\label{ch:conclusie}

% TODO: Trek een duidelijke conclusie, in de vorm van een antwoord op de
% onderzoeksvra(a)g(en). Wat was jouw bijdrage aan het onderzoeksdomein en
% hoe biedt dit meerwaarde aan het vakgebied/doelgroep? 
% Reflecteer kritisch over het resultaat. In Engelse teksten wordt deze sectie
% ``Discussion'' genoemd. Had je deze uitkomst verwacht? Zijn er zaken die nog
% niet duidelijk zijn?
% Heeft het onderzoek geleid tot nieuwe vragen die uitnodigen tot verder 
%onderzoek?

\lipsum[76-80]



%%=============================================================================
%% Bijlagen
%%=============================================================================

\appendix
\renewcommand{\chaptername}{Appendix}

%%---------- Onderzoeksvoorstel -----------------------------------------------

\chapter{Onderzoeksvoorstel}

Het onderwerp van deze bachelorproef is gebaseerd op een onderzoeksvoorstel dat vooraf werd beoordeeld door de promotor. Dat voorstel is opgenomen in deze bijlage.

% Verwijzing naar het bestand met de inhoud van het onderzoeksvoorstel
%---------- Inleiding ---------------------------------------------------------

\section{Introductie} % The \section*{} command stops section numbering
\label{sec:introductie}

Drones zijn sinds enkele jaren aan een grote opmars bezig, waarbij ze steeds meer worden ingezet voor niet-militaire doeleinden. Toen de orkaan Katrina in 2005 over de staten Louisiana en Mississippi in de Verenigde Staten trok werden drones voor het eerst ingezet in het rampgebied. \autocite{Murphy2015} Via de drones werd er op zoek gegaan naar overlevers tussen het puin en kon er ook meteen opgemeten worden wat de schade aan gebouwen, straten en steden was.

Dankzij de onmisbare hulp die de drones in deze levensbedreigende  situatie boden werden ze sindsdien alsmaar meer ingezet in rampgebieden. 

Enerzijds is de onderzoeksvraag in welke mate een drone kan ingezet worden bij een specifieke calamiteit. Anderzijds is er welke technische specificaties deze drones moeten hebben en welke beperkingen ze met zich meedragen. De technische specificaties bestaan dan vooral uit: beeld en geluid van de drone, het aantal drones die nodig zijn, probleem van batterijduur, hoe de drone wordt bestuurd; autonoom of manueel via een piloot...

%---------- Stand van zaken ---------------------------------------------------

\section{State-of-the-art}
\label{sec:state-of-the-art}

Een van de meest succesvolle instanties van drones die werden ingezet in een rampgebied is in het geval van Camp Fire in Californië in 2018. Dit was de meest dodelijke en destructieve bosbrand ooit in Californië. Hierbij zijn er bijna 100 doden gevallen en duizenden mensen verloren hun huis.

Drones werden hier meteen in grote hoeveelheid ingeze0t om het volledige gebied in kaart te brengen. Dit gebeurde door de drones te verdelen over gebieden en hiervan foto's te laten nemen die meteen toegevoegd werden aan een interactieve map. Uiteindelijk werden er van het rampgebied meer dan 70000 foto's genomen wat er voor zorgde dat er een zeer accurate map bestond van het getroffen gebied. \autocite{Reagan2019}

Een ander scenario waarin drones van groot belang waren was tijdens de orkaan Irma in de Caraïben in 2017. Ook hier werd een grote hoeveelheid drones ingezet om het verwoeste gebied in kaart te brengen. \autocite{Morant}

Hierbij heeft Dronotec, een bedrijf dat zich specialiseert in drone inspecties voor verzekeringsbedrijven, drones uitgestuurd om de schade aan 300 gebouwen in 10 dagen op te meten. Op deze manier kon dit veel veiliger en sneller gebeuren dan op een conventionele wijze.
% Voor literatuurverwijzingen zijn er twee belangrijke commando's:
% \autocite{KEY} => (Auteur, jaartal) Gebruik dit als de naam van de auteur
%   geen onderdeel is van de zin.
% \textcite{KEY} => Auteur (jaartal)  Gebruik dit als de auteursnaam wel een
%   functie heeft in de zin (bv. ``Uit onderzoek door Doll & Hill (1954) bleek
%   ...'')


%---------- Methodologie ------------------------------------------------------
\section{Methodologie}
\label{sec:methodologie}

Om een antwoord te vinden op de eerste onderzoeksvraag zal er eerst en vooral gekeken worden naar een groot aantal calamiteiten en op welke manieren een drone hierbij een verschil zou kunnen maken. Dit zal gebeuren door mogelijke oplossingen te vinden voor problemen die een specifieke ramp met zich meebrengt. 

Om de tweede onderzoeksvraag te beantwoorden zal er vooral gekeken worden naar de technische aspecten van een drone, welke zaken beperkingen met zich meebrengen en hoe die eventueel kunnen opgelost of vermeden worden.

Dit zal voor beide vragen aan de hand gebeuren van interviews, artikels, boeken en specificaties van drones.

%---------- Verwachte resultaten ----------------------------------------------
\section{Verwachte resultaten}
\label{sec:verwachte_resultaten}

Concrete resultaten zijn in het geval van dit onderzoek moeilijk in kaart te brengen. Het eigenlijke onderzoek naar deze onderzoeksvragen kan niet fysiek gebeuren of in realiteit worden getest. Er kan wel vergeleken worden met werkelijke resultaten die alreeds zijn behaald en hieruit conclusies te trekken of dit op andere gebieden ook toepasbaar is. 

%---------- Verwachte conclusies ----------------------------------------------
\section{Verwachte conclusies}
\label{sec:verwachte_conclusies}

Er wordt verwacht dat uit het onderzoek van de twee onderzoeksvragen er een goed beeld zal gevormd worden in hoeverre een drone echt wel een verschil kan maken in een bepaalde ramp en op welke manier dit zal gebeuren. Anderzijds zal er ook een realistisch weergave zijn van waarover de drone allemaal moet beschikken om dit te kunnen realiseren. Er wordt ook oplossing geboden voor het feit dat drones over het algemeen over een korte batterijduur beschikken.



%%---------- Andere bijlagen --------------------------------------------------
% TODO: Voeg hier eventuele andere bijlagen toe
%\input{...}

%%---------- Referentielijst --------------------------------------------------

\printbibliography[heading=bibintoc]

\end{document}
