%==============================================================================
% Sjabloon onderzoeksvoorstel bachelorproef
%==============================================================================
% Gebaseerd op LaTeX-sjabloon ‘Stylish Article’ (zie voorstel.cls)
% Auteur: Jens Buysse, Bert Van Vreckem
%
% Compileren in TeXstudio:
%
% - Zorg dat Biber de bibliografie compileert (en niet Biblatex)
%   Options > Configure > Build > Default Bibliography Tool: "txs:///biber"
% - F5 om te compileren en het resultaat te bekijken.
% - Als de bibliografie niet zichtbaar is, probeer dan F5 - F8 - F5
%   Met F8 compileer je de bibliografie apart.
%
% Als je JabRef gebruikt voor het bijhouden van de bibliografie, zorg dan
% dat je in ``biblatex''-modus opslaat: File > Switch to BibLaTeX mode.

\documentclass{voorstel}

\usepackage{lipsum}

%------------------------------------------------------------------------------
% Metadata over het voorstel
%------------------------------------------------------------------------------

%---------- Titel & auteur ----------------------------------------------------

% TODO: geef werktitel van je eigen voorstel op
\PaperTitle{Drones bij calamiteiten}
\PaperType{Onderzoeksvoorstel Bachelorproef 2019-2020} % Type document

% TODO: vul je eigen naam in als auteur, geef ook je emailadres mee!
\Authors{Beau Bellon\textsuperscript{1}} % Authors
\CoPromotor{Nog niet aangewezen\textsuperscript{2}}
\affiliation{\textbf{Contact:}
  \textsuperscript{1} \href{mailto:beau.bellon@student.hogent.be}{beau.bellon@student.hogent.be};
  \textsuperscript{2} \href{mailto:}{};
}

%---------- Abstract ----------------------------------------------------------

\Abstract{Drones zijn al sinds enkele decennia ingezet met vooral militaire doeleinden, maar sinds enkele jaren hebben deze ook hun plaats gevonden voor maatschappelijke doelen. In dit onderzoek wordt er op zoek gegaan of drones bij verschillende soorten calamiteiten ingezet kunnen worden om sneller hulpverlening te bieden, schade op te meten of ze in het algemeen een verschil kunnen maken. Anderzijds gaan de technische specificaties die nodig zijn ook uitbundig besproken worden. Er wordt gekeken naar de nodige zaken waarover de drone moet beschikken om bij een specifieke ramp hulp te kunnen bieden. In een levensbedreigende situatie is het vaak zo dat elke seconde telt en dan zou de hulp van een drone, autonoom of manueel, wel eens het verschil kunnen maken. Het doel van dit onderzoek is dan ook om na te gaan of een drone effectief in verschillende rampgebieden kan worden ingezet om hulp te verlenen. Ook worden de technische beperkingen onderzocht en in kaart gebracht.

}

%---------- Onderzoeksdomein en sleutelwoorden --------------------------------
% TODO: Sleutelwoorden:
%
% Het eerste sleutelwoord beschrijft het onderzoeksdomein. Je kan kiezen uit
% deze lijst:
%
% - Mobiele applicatieontwikkeling
% - Webapplicatieontwikkeling
% - Applicatieontwikkeling (andere)
% - Systeembeheer
% - Netwerkbeheer
% - Mainframe
% - E-business
% - Databanken en big data
% - Machineleertechnieken en kunstmatige intelligentie
% - Andere (specifieer)
%
% De andere sleutelwoorden zijn vrij te kiezen

\Keywords{Studie. Drones --- Hulpverlening --- Technische specificaties } % Keywords
\newcommand{\keywordname}{Sleutelwoorden} % Defines the keywords heading name

%---------- Titel, inhoud -----------------------------------------------------

\begin{document}

\flushbottom % Makes all text pages the same height
\maketitle % Print the title and abstract box
\tableofcontents % Print the contents section
\thispagestyle{empty} % Removes page numbering from the first page

%------------------------------------------------------------------------------
% Hoofdtekst
%------------------------------------------------------------------------------

% De hoofdtekst van het voorstel zit in een apart bestand, zodat het makkelijk
% kan opgenomen worden in de bijlagen van de bachelorproef zelf.
%---------- Inleiding ---------------------------------------------------------

\section{Introductie} % The \section*{} command stops section numbering
\label{sec:introductie}

Drones zijn sinds enkele jaren aan een grote opmars bezig, waarbij ze steeds meer worden ingezet voor niet-militaire doeleinden. Toen de orkaan Katrina in 2005 over de staten Louisiana en Mississippi in de Verenigde Staten trok werden drones voor het eerst ingezet in het rampgebied. \autocite{Murphy2015} Via de drones werd er op zoek gegaan naar overlevers tussen het puin en kon er ook meteen opgemeten worden wat de schade aan gebouwen, straten en steden was.

Dankzij de onmisbare hulp die de drones in deze levensbedreigende  situatie boden werden ze sindsdien alsmaar meer ingezet in rampgebieden. 

Enerzijds is de onderzoeksvraag in welke mate een drone kan ingezet worden bij een specifieke calamiteit. Anderzijds is er welke technische specificaties deze drones moeten hebben en welke beperkingen ze met zich meedragen. De technische specificaties bestaan dan vooral uit: beeld en geluid van de drone, het aantal drones die nodig zijn, probleem van batterijduur, hoe de drone wordt bestuurd; autonoom of manueel via een piloot...

%---------- Stand van zaken ---------------------------------------------------

\section{State-of-the-art}
\label{sec:state-of-the-art}

Een van de meest succesvolle instanties van drones die werden ingezet in een rampgebied is in het geval van Camp Fire in Californië in 2018. Dit was de meest dodelijke en destructieve bosbrand ooit in Californië. Hierbij zijn er bijna 100 doden gevallen en duizenden mensen verloren hun huis.

Drones werden hier meteen in grote hoeveelheid ingeze0t om het volledige gebied in kaart te brengen. Dit gebeurde door de drones te verdelen over gebieden en hiervan foto's te laten nemen die meteen toegevoegd werden aan een interactieve map. Uiteindelijk werden er van het rampgebied meer dan 70000 foto's genomen wat er voor zorgde dat er een zeer accurate map bestond van het getroffen gebied. \autocite{Reagan2019}

Een ander scenario waarin drones van groot belang waren was tijdens de orkaan Irma in de Caraïben in 2017. Ook hier werd een grote hoeveelheid drones ingezet om het verwoeste gebied in kaart te brengen. \autocite{Morant}

Hierbij heeft Dronotec, een bedrijf dat zich specialiseert in drone inspecties voor verzekeringsbedrijven, drones uitgestuurd om de schade aan 300 gebouwen in 10 dagen op te meten. Op deze manier kon dit veel veiliger en sneller gebeuren dan op een conventionele wijze.
% Voor literatuurverwijzingen zijn er twee belangrijke commando's:
% \autocite{KEY} => (Auteur, jaartal) Gebruik dit als de naam van de auteur
%   geen onderdeel is van de zin.
% \textcite{KEY} => Auteur (jaartal)  Gebruik dit als de auteursnaam wel een
%   functie heeft in de zin (bv. ``Uit onderzoek door Doll & Hill (1954) bleek
%   ...'')


%---------- Methodologie ------------------------------------------------------
\section{Methodologie}
\label{sec:methodologie}

Om een antwoord te vinden op de eerste onderzoeksvraag zal er eerst en vooral gekeken worden naar een groot aantal calamiteiten en op welke manieren een drone hierbij een verschil zou kunnen maken. Dit zal gebeuren door mogelijke oplossingen te vinden voor problemen die een specifieke ramp met zich meebrengt. 

Om de tweede onderzoeksvraag te beantwoorden zal er vooral gekeken worden naar de technische aspecten van een drone, welke zaken beperkingen met zich meebrengen en hoe die eventueel kunnen opgelost of vermeden worden.

Dit zal voor beide vragen aan de hand gebeuren van interviews, artikels, boeken en specificaties van drones.

%---------- Verwachte resultaten ----------------------------------------------
\section{Verwachte resultaten}
\label{sec:verwachte_resultaten}

Concrete resultaten zijn in het geval van dit onderzoek moeilijk in kaart te brengen. Het eigenlijke onderzoek naar deze onderzoeksvragen kan niet fysiek gebeuren of in realiteit worden getest. Er kan wel vergeleken worden met werkelijke resultaten die alreeds zijn behaald en hieruit conclusies te trekken of dit op andere gebieden ook toepasbaar is. 

%---------- Verwachte conclusies ----------------------------------------------
\section{Verwachte conclusies}
\label{sec:verwachte_conclusies}

Er wordt verwacht dat uit het onderzoek van de twee onderzoeksvragen er een goed beeld zal gevormd worden in hoeverre een drone echt wel een verschil kan maken in een bepaalde ramp en op welke manier dit zal gebeuren. Anderzijds zal er ook een realistisch weergave zijn van waarover de drone allemaal moet beschikken om dit te kunnen realiseren. Er wordt ook oplossing geboden voor het feit dat drones over het algemeen over een korte batterijduur beschikken.



%------------------------------------------------------------------------------
% Referentielijst
%------------------------------------------------------------------------------
% TODO: de gerefereerde werken moeten in BibTeX-bestand ``voorstel.bib''
% voorkomen. Gebruik JabRef om je bibliografie bij te houden en vergeet niet
% om compatibiliteit met Biber/BibLaTeX aan te zetten (File > Switch to
% BibLaTeX mode)

\phantomsection
\printbibliography[heading=bibintoc]

\end{document}
