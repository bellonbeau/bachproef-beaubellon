%---------- Inleiding ---------------------------------------------------------

\section{Introductie} % The \section*{} command stops section numbering
\label{sec:introductie}

Drones zijn sinds enkele jaren aan een grote opmars bezig, waarbij ze steeds meer worden ingezet voor niet-militaire doeleinden. Toen de orkaan Katrina in 2005 over de staten Louisiana en Mississippi in de Verenigde Staten trok werden drones voor het eerst ingezet in het rampgebied. \autocite{Murphy2015} Via de drones werd er op zoek gegaan naar overlevers tussen het puin en kon er ook meteen opgemeten worden wat de schade aan gebouwen, straten en steden was. Omdat deze gebieden door de schade moeilijk bereikbaar waren op conventionele wijze werd er via het gebruik van drones genavigeerd door het puin en werden er op deze manier foto's en video's gemaakt. Zo was er meteen een beter beeld wat de schade aan bruggen, huizen en straten was.


Het is dankzij deze eerste succesvolle inzetting van drones dat deze steeds meer werden gebruikt tijdens levensbedreigende situaties om hulpverlening toe te kennen. Tijdens de bosbranden in Australië zijn drones ook ingezet om mee te helpen met branden te blussen, foto's te maken van de getroffen gebieden en identificeren van dieren en planten.

De onderzoeksvraag houdt in op welke manier drones tijdens en na de bosbranden in Australië hulpverlening hebben kunnen bieden, op welke manier dit is gebeurt en wat er moet gebeuren met de data die de drone over het rampgebied verzamelt. Anderzijds is er welke technische specificaties deze drones moeten hebben en welke beperkingen ze met zich meedragen. De technische specificaties bestaan dan vooral uit: beeld en geluid van de drone, het aantal drones die nodig zijn, hoe de drone wordt bestuurd; autonoom of manueel via een piloot, wat er met de data die verzamelt wordt moet gebeuren.

%---------- Stand van zaken ---------------------------------------------------

\section{State-of-the-art}
\label{sec:state-of-the-art}

Een van de meest succesvolle instanties van drones die werden ingezet in een rampgebied is in het geval van Camp Fire in Californië in 2018. Dit was de meest dodelijke en destructieve bosbrand ooit in Californië. Hierbij zijn er bijna 100 doden gevallen en duizenden mensen verloren hun huis.

Drones werden hier meteen in grote hoeveelheid ingezet om het volledige gebied in kaart te brengen door middel van Image Recognition. Dit gebeurde door de drones te verdelen over gebieden en hiervan foto's te laten nemen die meteen toegevoegd werden aan een interactieve map. Uiteindelijk werden er van het rampgebied meer dan 70000 foto's genomen wat er voor zorgde dat er een zeer accurate map bestond van het getroffen gebied. \autocite{Reagan2019}

Een ander scenario waarin drones van groot belang waren, was tijdens de orkaan Irma in de Caraïben in 2017. Ook hier werd een grote hoeveelheid drones ingezet om het verwoeste gebied in kaart te brengen. \autocite{Morant}

Hierbij heeft Dronotec, een bedrijf dat zich specialiseert in drone inspecties voor verzekeringsbedrijven, drones uitgestuurd om de schade aan 300 gebouwen in 10 dagen op te meten. Op deze manier kon dit veel veiliger en sneller gebeuren dan op een conventionele wijze.
% Voor literatuurverwijzingen zijn er twee belangrijke commando's:
% \autocite{KEY} => (Auteur, jaartal) Gebruik dit als de naam van de auteur
%   geen onderdeel is van de zin.
% \textcite{KEY} => Auteur (jaartal)  Gebruik dit als de auteursnaam wel een
%   functie heeft in de zin (bv. ``Uit onderzoek door Doll & Hill (1954) bleek
%   ...'')


%---------- Methodologie ------------------------------------------------------
\section{Methodologie}
\label{sec:methodologie}

Om een antwoord te vinden op de onderzoeksvraag zal er eerst en vooral onderzocht worden op welke manier drones werden ingezet tijdens en na deze bosbranden. Dit zal voornamelijk gaan over het helpen bij het blussen, het maken van een interactieve map via foto's en video's en het herkennen van (bedreigde) dieren en plantensoorten.

Omdat het niet enkel belangrijk is om te weten op welke manier dit is gebeurt, zal er ook grondig onderzocht worden wat er met de data die verzamelt wordt gebeurt en op welke manier deze verzamelt wordt. Er zal hierbij gekeken worden naar de technieken die door de drones werden toegepast, het soort data dat zij in een specifieke situatie verzamelden en de technische specificaties en beperkingen waarover ze in deze situaties beschikken. 

In elk van deze gevallen zal dus goed onderzocht worden wat de effectieve functionaliteiten van de drone moeten zijn en op welke manier zij beschermd worden tegen misbruik van buitenaf. Ook wordt er onderzocht of dit in sommige gevallen beter had kunnen verlopen.

Het beantwoorden van de onderzoeksvraag zal gebeuren aan de hand van interviews, artikels, blogs, nieuwsberichten, het onderzoeken op welke manier drones data verwerken en op welke manier zij beschermd zijn.

%---------- Verwachte resultaten ----------------------------------------------
\section{Verwachte resultaten}
\label{sec:verwachte_resultaten}

Het resultaat van dit onderzoek zal aanwijzen op welke manier drones hulpverlening hebben toegekend tijdens de bosbranden in Australië, op welke manier dit is gebeurt en hoe de data verwerking in zijn gang gaat. Er zal een duidelijk beeld zijn van de effectieve hulp die is aangeboden via drones en de technische aspecten zullen goed in kaart worden gebracht. Ook wordt er gereflecteerd op welke manieren drones nog een grotere rol zouden kunnen gespeeld hebben.


%---------- Verwachte conclusies ----------------------------------------------
\section{Verwachte conclusies}
\label{sec:verwachte_conclusies}

Er zal gereflecteerd worden of alles goed is verlopen en in welke mate drones een impact hebben gehad op het in kaart brengen van de getroffen gebieden. Er zal een goed beeld zijn van wat er achter de schermen allemaal moet gebeuren om de data die de drone verzamelt te verwerken en op welke wijze dit gebeurt.

De technische beperkingen waarover een drone beschikt zullen aan bod komen en er wordt op zoek gegaan naar een realistische oplossing voor deze beperkingen.

